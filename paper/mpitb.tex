% !TeX root = RJwrapper.tex
\title{mpitb: A toolbox for calculating multidimensional poverty indices in R}


\author{by Ignacio Girela}

\maketitle

\abstract{%
This article presents \pkg{mpitb}, an R package for calculating multidimensional poverty indices (MPIs) based on the popular Alkire-Foster (AF) measurement approach. \pkg{mpitb} package provides a tractable and extensive framework for researchers, analysts, and practitioners working on mutltidimensional poverty measurement projects in the same vein of the well-known global MPI workflow. The toolbox mainly consists of providing methods for estimating AF measures such that they can be easily manipulated. In this way, users can concentrate on the very analysis of results. Furthermore, this package accounts for the complex survey design of micro data which is of key relevance in statistical inference of the estimates. A demonstration about the usage of \pkg{mpitb} package with a synthetic data which has a typical household survey design.
}

\hypertarget{introduction}{%
\section{Introduction}\label{introduction}}

Although poverty is widely understood as the inability to basic living standards multidimensional well-being status, the measurement of poverty has traditionally been based on monetary deficits. Over the past two decades, there has been a significant increase in the use of multidimensional approaches to measuring poverty in an attempt to capture the complexity of this phenomenon. In this context, the dual-cut-off-counting measurement approach proposed by Alkire and Foster (2011) (AF, hereafter) has become particularly popular due to its flexibility and ability to inform policy. For example, the AF method is the basis of the global Multidimensional Poverty Index (MPI) which is yearly published by the United Nations Development Programme (UNDP) and the Oxford Poverty and Human Development Initiative (OPHI) (UNDP and OPHI 2022). In addition, many countries are developing their own official MPIs to track poverty reduction progress using the AF method.

This measurement approach stands out both for providing an extensive policy information platform and remaining relatively understandable by the general public. In other words, AF measures point estimates entail simple algebraic operations. Some R packages that computes AF measures include \CRANpkg{MPI} (Kukiattikun and Chainarong 2022) and \CRANpkg{mpindex} (Abdulsamad 2023). Notwithstanding, these packages do not account for the complex survey design of micro data and, as a result, assume the data was obtained through a simple random sampling. Rarely is this the case of household surveys. Not considering the complex survey design affects further statistical inference exercises of interest. Namely, is multidimensional poverty in one subgroup of the population greater than another? Has poverty been reduced over time?

Therefore, calculating AF measures in practice is still a challenging task. For this reason OPHI launched a Stata package \texttt{mpitb} with the aim of providing a integrated framework that mirrors the estimation process of the well-known global MPI to researchers, analysts and practitioners (Suppa 2023). In a nutshell, this package provides a set of subcommands for estimating key quantities. Not only AF measures but also their standard errors (considering sampling design) and confidence intervals. These quantities are not usually reported in poverty studies, however, they are for statistical inference. Thus, users can focus on the analysis of results.

With the aim of adapting this framework for R users, I developed the \CRANpkg{mpitb} package.

The remainder of this paper is organized as follows.

\hypertarget{measuring-multidimensional-poverty-the-alkire-foster-method}{%
\section{Measuring multidimensional poverty: the Alkire-Foster Method}\label{measuring-multidimensional-poverty-the-alkire-foster-method}}

Alkire and Foster (2011) proposed a flexible approach to measure multidimensional poverty that can be tailored to different contexts and policy purposes. Throughout an multidimensional povery index (MPI) construction process it is possible to distinguish the \textbf{identification} and \textbf{aggregation} steps (for a detailed description, see Alkire et al. 2015). Flexibility feature of this method derived from the so-called ``dual cutoff counting approach'' for identifying the poor and the possibility to build the MPI by aggregating different partial measures.

\hypertarget{identification-step}{%
\subsubsection{Identification step}\label{identification-step}}

\begin{enumerate}
\item Determine a set of dimensions of poverty $\mathcal{D}$ that are considered relevant for human development in a specific context (e.g., the global MPI chooses dimensions of health, education, and living standards, but other dimensions can be chosen depending on the context and goals).
\item Select $d$ indicators that represents deprivations in each dimension (e.g., child mortality and malnutrition are the two indicators that represents health dimension in the global MPI).  
\item Assign weights to each dimension and indicator, reflecting their relative importance where $w_j$ is represents the weight of the $j$-th indicator for $j = 1,\ldots,d$. In practice, indicators in each dimension are weighted equally such that $\sum_{j=1}^d w_j = 1$.
\item Set the indicators deprivation cutoffs, which define the minimum level of achievement required to be considered non-deprived in each indicator (e.g., the global MPI uses cutoffs of having at least five years of schooling, having access to electricity, etc., but different thresholds can be set reflecting desired standards). 
\item Apply the deprivations cutoff vector to each of the $n$ observations (individuals or households) and build the $n \times d$ deprivation matrix $\mathbf{g^0}$. Each element $\mathbf{g^0}_{ij}$ of this matrix is a binary variable. If $\mathbf{g^0}_{ij} = 1$, the $i$-th observation is deprived in indicator $j$, and the opposite if $\mathbf{g^0}_{ij} = 0$. 
\item Build the weighted deprivation matrix assigning the corresponding weight to each indicator and calculate the deprivations score for each observation $c_i$, which is the weighted sum of the deprivations $c_i = \sum_{j=1}^d w_j d_ij$.
\item Identify who is poor by setting a unique poverty cutoff $k$ meaning the minimum proportion of weighted deprivations a household needs to experience to be considered multidimensional poor. This cutoff is compared with the deprivations score. Therefore, if $c_i \geq k$, the person is multidimensional poor (e.g., the global MPI uses a cutoff $k$ of 33.3\% or 1/3 which means that a person is poor if it is deprived in one-third or more of the weighted indicators).
\item Censor data of the non-poor and get the so-called censored (weighted) deprivation matrix ($\mathbf{\bar{g}^0}(k)$), and censored deprivations scores ($c(k)$), where $c_i(k) = c_i$ if $c_i \geq k $, and $c_i(k) = 0$ otherwise.
\end{enumerate}

\hypertarget{aggregation-step}{%
\subsubsection{Aggregation step}\label{aggregation-step}}

There are two different aggregation procedures from matrix \(\mathbf{\bar{g}^0}(k)\) depending on the order of aggregation (by rows and then by columnss o fir by columns and then by rows.)

\begin{enumerate}
\item Compute the MPI, also known as Adjusted Headcount Ratio ($M_0$), by taking the mean of the censored deprivation score ($c_(k)$):
\[M_0 = \sum_{i=1}^n c_i(k) \]
\end{enumerate}

\hypertarget{illustrative-example}{%
\subsubsection{Illustrative example}\label{illustrative-example}}

Assume that we a sample of 5 people and we set 3 dimensions (health, education and living standards). The first dimension is represented by an indicator of nutrition (N). The second by years of schooling (YS) and the third by a housing quality indicator (H) and having access to drinkable water (W).
\[ \mathbf{g^0} = 
\begin{blockarray}{cccc}
N & YS & H & W \\
\begin{block}{(cccc)}
  1 & 1 & 0 & 1 \\
  0 & 1 & 0 & 0 \\
  0 & 0 & 1 & 0 \\
  0 & 1 & 0 & 1 \\
  0 & 0 & 0 & 0 \\
\end{block}
\end{blockarray}
\]

In this case, the first person is deprived simultaneously in N, YS and W indicators.

In our latter example, assume that all dimensions are equally weighted (a weight of \(1/3\)) and within each dimension indicators have proportionally equal weights (i.e., N and YS weight \(1/3\), whereas H and W \(1/6\).)

\[ \mathbf{\bar{g}^0} = 
\begin{blockarray}{ccccc}
N & YS & H & W & \bf{c}\\
\begin{block}{(cccc)c}
  1/3 & 1/3 & 0 & 1/6 & 5/6\\
  0 & 1/3 & 0 & 0 & 1/3\\
  0 & 0 & 1/6 & 0 & 1/6\\
  0 & 1/3 & 0 & 1/6 & 3/6\\
  0 & 0 & 0 & 0 & 0\\
\end{block}
  a & b & c & d & 
\end{blockarray}
\]
In our example, assume that we set \(k = 1/3\). Hence,

\[ \mathbf{\bar{g}^0}(k) = 
\begin{blockarray}{ccccc}
N & YS & H & W & \bf{c(k)}\\
\begin{block}{(cccc)c}
  1/3 & 1/3 & 0 & 1/6 & 5/6\\
  0 & 1/3 & 0 & 0 & 1/3\\
  0 & 0 & 0 & 0 & 0\\
  0 & 1/3 & 0 & 1/6 & 3/6\\
  0 & 0 & 0 & 0 & 0\\
\end{block}
\end{blockarray}
\]

Note that the third and fifth person are censored from the analysis and three people are identified as poor.

(continue \ldots)

(continue \ldots)

\hypertarget{overview-of-package}{%
\section{\texorpdfstring{Overview of \pkg{mpitb} package}{Overview of  package}}\label{overview-of-package}}

\pkg{mpitb} is a packages for \ldots{}

\hypertarget{applications}{%
\section{Applications}\label{applications}}

Suppa (2023) provide a set of examples on a synthetic data set following a common household survey structure. With the view of consistency, all applied examples here below are based on the same data.

\hypertarget{estimate-af-measures-for-a-single-county}{%
\subsection{Estimate AF measures for a single county}\label{estimate-af-measures-for-a-single-county}}

\begin{verbatim}
library(mpitb)

data <- syn_cdta

data <- subset(data, t == 1)
\end{verbatim}

\hypertarget{r-penguins-tab-static-eval-knitris_latex_output-knitrkableheadpenguins-format-latex-caption-a-basic-table-kableextrakable_stylingfont_size-7}{%
\section{\texorpdfstring{\texttt{\{r\ penguins-tab-static,\ eval\ =\ knitr::is\_latex\_output()\}\ \#\ knitr::kable(head(penguins),\ format\ =\ "latex",\ caption\ =\ "A\ basic\ table")\ \%\textgreater{}\%\ \ \#\ \ \ kableExtra::kable\_styling(font\_size\ =\ 7)\ \#}}{\{r penguins-tab-static, eval = knitr::is\_latex\_output()\} \# knitr::kable(head(penguins), format = "latex", caption = "A basic table") \%\textgreater\%  \#   kableExtra::kable\_styling(font\_size = 7) \#}}\label{r-penguins-tab-static-eval-knitris_latex_output-knitrkableheadpenguins-format-latex-caption-a-basic-table-kableextrakable_stylingfont_size-7}}

\hypertarget{r-penguins-plotly-echo-true-fig.height-5-fig.capa-basic-interactive-plot-made-with-the-plotly-package-on-palmer-penguin-data.-three-species-of-penguins-are-plotted-with-bill-depth-on-the-x-axis-and-bill-length-on-the-y-axis.-when-hovering-on-a-point-a-tooltip-will-show-the-exact-value-of-the-bill-depth-and-length-for-that-point-along-with-the-species-name.-includeknitris_html_output-evalknitris_html_output-fig.alt-a-scatterplot-of-bill-length-against-bill-depth-both-measured-in-millimetre.-the-three-species-are-shown-in-different-colours-and-loosely-forms-three-clusters.-adelie-has-small-bill-length-and-large-bill-depth-gentoo-has-small-bill-depth-but-large-bill-length-and-chinstrap-has-relatively-large-bill-depth-and-bill-length.-p---penguins-ggplotaesx-bill_depth_mm-y-bill_length_mm-color-species-geom_point-ggplotlyp}{%
\section{\texorpdfstring{\texttt{\{r\ penguins-plotly,\ echo\ =\ TRUE,\ fig.height\ =\ 5,\ fig.cap="A\ basic\ interactive\ plot\ made\ with\ the\ plotly\ package\ on\ palmer\ penguin\ data.\ Three\ species\ of\ penguins\ are\ plotted\ with\ bill\ depth\ on\ the\ x-axis\ and\ bill\ length\ on\ the\ y-axis.\ When\ hovering\ on\ a\ point,\ a\ tooltip\ will\ show\ the\ exact\ value\ of\ the\ bill\ depth\ and\ length\ for\ that\ point,\ along\ with\ the\ species\ name.",\ include=knitr::is\_html\_output(),\ eval=knitr::is\_html\_output(),\ fig.alt\ =\ "A\ scatterplot\ of\ bill\ length\ against\ bill\ depth,\ both\ measured\ in\ millimetre.\ The\ three\ species\ are\ shown\ in\ different\ colours\ and\ loosely\ forms\ three\ clusters.\ Adelie\ has\ small\ bill\ length\ and\ large\ bill\ depth,\ Gentoo\ has\ small\ bill\ depth\ but\ large\ bill\ length,\ and\ Chinstrap\ has\ relatively\ large\ bill\ depth\ and\ bill\ length."\}\ \#\ p\ \textless{}-\ penguins\ \%\textgreater{}\%\ \ \#\ \ \ ggplot(aes(x\ =\ bill\_depth\_mm,\ y\ =\ bill\_length\_mm,\ \ \#\ \ \ \ \ \ \ \ \ \ \ \ \ \ color\ =\ species))\ +\ \ \#\ \ \ geom\_point()\ \#\ ggplotly(p)\ \#}}{\{r penguins-plotly, echo = TRUE, fig.height = 5, fig.cap="A basic interactive plot made with the plotly package on palmer penguin data. Three species of penguins are plotted with bill depth on the x-axis and bill length on the y-axis. When hovering on a point, a tooltip will show the exact value of the bill depth and length for that point, along with the species name.", include=knitr::is\_html\_output(), eval=knitr::is\_html\_output(), fig.alt = "A scatterplot of bill length against bill depth, both measured in millimetre. The three species are shown in different colours and loosely forms three clusters. Adelie has small bill length and large bill depth, Gentoo has small bill depth but large bill length, and Chinstrap has relatively large bill depth and bill length."\} \# p \textless- penguins \%\textgreater\%  \#   ggplot(aes(x = bill\_depth\_mm, y = bill\_length\_mm,  \#              color = species)) +  \#   geom\_point() \# ggplotly(p) \#}}\label{r-penguins-plotly-echo-true-fig.height-5-fig.capa-basic-interactive-plot-made-with-the-plotly-package-on-palmer-penguin-data.-three-species-of-penguins-are-plotted-with-bill-depth-on-the-x-axis-and-bill-length-on-the-y-axis.-when-hovering-on-a-point-a-tooltip-will-show-the-exact-value-of-the-bill-depth-and-length-for-that-point-along-with-the-species-name.-includeknitris_html_output-evalknitris_html_output-fig.alt-a-scatterplot-of-bill-length-against-bill-depth-both-measured-in-millimetre.-the-three-species-are-shown-in-different-colours-and-loosely-forms-three-clusters.-adelie-has-small-bill-length-and-large-bill-depth-gentoo-has-small-bill-depth-but-large-bill-length-and-chinstrap-has-relatively-large-bill-depth-and-bill-length.-p---penguins-ggplotaesx-bill_depth_mm-y-bill_length_mm-color-species-geom_point-ggplotlyp}}

\hypertarget{r-penguins-ggplot-echo-true-fig.height-5-fig.capa-basic-non-interactive-plot-made-with-the-ggplot2-package-on-palmer-penguin-data.-three-species-of-penguins-are-plotted-with-bill-depth-on-the-x-axis-and-bill-length-on-the-y-axis.-visit-the-online-article-to-access-the-interactive-version-made-with-the-plotly-package.-includeknitris_latex_output-evalknitris_latex_output-penguins-ggplotaesx-bill_depth_mm-y-bill_length_mm-color-species-geom_point}{%
\section{\texorpdfstring{\texttt{\{r\ penguins-ggplot,\ echo\ =\ TRUE,\ fig.height\ =\ 5,\ fig.cap="A\ basic\ non-interactive\ plot\ made\ with\ the\ ggplot2\ package\ on\ palmer\ penguin\ data.\ Three\ species\ of\ penguins\ are\ plotted\ with\ bill\ depth\ on\ the\ x-axis\ and\ bill\ length\ on\ the\ y-axis.\ Visit\ the\ online\ article\ to\ access\ the\ interactive\ version\ made\ with\ the\ plotly\ package.",\ include=knitr::is\_latex\_output(),\ eval=knitr::is\_latex\_output()\}\ \#\ penguins\ \%\textgreater{}\%\ \ \#\ \ \ ggplot(aes(x\ =\ bill\_depth\_mm,\ y\ =\ bill\_length\_mm,\ \ \#\ \ \ \ \ \ \ \ \ \ \ \ \ \ color\ =\ species))\ +\ \ \#\ \ \ geom\_point()\ \#}}{\{r penguins-ggplot, echo = TRUE, fig.height = 5, fig.cap="A basic non-interactive plot made with the ggplot2 package on palmer penguin data. Three species of penguins are plotted with bill depth on the x-axis and bill length on the y-axis. Visit the online article to access the interactive version made with the plotly package.", include=knitr::is\_latex\_output(), eval=knitr::is\_latex\_output()\} \# penguins \%\textgreater\%  \#   ggplot(aes(x = bill\_depth\_mm, y = bill\_length\_mm,  \#              color = species)) +  \#   geom\_point() \#}}\label{r-penguins-ggplot-echo-true-fig.height-5-fig.capa-basic-non-interactive-plot-made-with-the-ggplot2-package-on-palmer-penguin-data.-three-species-of-penguins-are-plotted-with-bill-depth-on-the-x-axis-and-bill-length-on-the-y-axis.-visit-the-online-article-to-access-the-interactive-version-made-with-the-plotly-package.-includeknitris_latex_output-evalknitris_latex_output-penguins-ggplotaesx-bill_depth_mm-y-bill_length_mm-color-species-geom_point}}

\hypertarget{summary}{%
\section{Summary}\label{summary}}

In this paper, we introduced a package for calculating multidimensional poverty indices based on the Alkire-Foster method

\hypertarget{acknowledgements}{%
\section{Acknowledgements}\label{acknowledgements}}

I would like to express my sincere thanks both to Rodrigo García Arancibia and José Vargas, my mentors, for their invaluable guidance, feedback, and encouragement throughout the development of this package. I also wish to acknowledge Nicolai Suppa for his generous support and enthusiasm in adapting his Stata package to R users. Without his constructive suggestions this package would not have been possible. Finally, I would like to thank every instructor from the OPHI Summer School 2022, for their thorough lectures on the Alkire-Foster method for multidimensional poverty measurement and the provision of the Stata source code which mainly motivated this project.

\hypertarget{references}{%
\section*{References}\label{references}}
\addcontentsline{toc}{section}{References}

\hypertarget{refs}{}
\begin{CSLReferences}{1}{0}
\leavevmode\vadjust pre{\hypertarget{ref-mpindexpkg}{}}%
Abdulsamad, Bhas. 2023. \emph{{m}pindex: Multidimensional Poverty Index (MPI)}. \url{https://CRAN.R-project.org/package=mpindex}.

\leavevmode\vadjust pre{\hypertarget{ref-af11}{}}%
Alkire, Sabina, and J. E. Foster. 2011. {``Counting and Multidimensional Poverty Measurement.''} \emph{Journal of Public Economics} 95 (7): 476--87.

\leavevmode\vadjust pre{\hypertarget{ref-alkire2015multidimensional}{}}%
Alkire, Sabina, James Foster, Suman Seth, Maria Emma Santos, José Manuel Roche, and Paola Ballon. 2015. \emph{Multidimensional Poverty Measurement and Analysis}. Oxford University Press.

\leavevmode\vadjust pre{\hypertarget{ref-MPIpkg}{}}%
Kukiattikun, Kittiya, and Chainarong Chainarong. 2022. \emph{MPI: Computation of Multidimensional Poverty Index (MPI)}. \url{https://CRAN.R-project.org/package=MPI}.

\leavevmode\vadjust pre{\hypertarget{ref-mpitb_Stata}{}}%
Suppa, Nicolai. 2023. {``{m}pitb: A Toolbox for Multidimensional Poverty Indices.''} \emph{The Stata Journal} 23 (3): 625--57. \url{https://doi.org/10.1177/1536867X231195286}.

\leavevmode\vadjust pre{\hypertarget{ref-gMPI-report2022}{}}%
UNDP, and OPHI. 2022. {``\href{}{2022 Global Multidimensional Poverty Index (MPI)}.''} \emph{UNDP (United Nations Development Programme)}.

\end{CSLReferences}

\bibliography{RJreferences.bib}

\address{%
Ignacio Girela\\
CONICET - Universidad Nacional de Córdoba\\%
Facultad de Ciencias Económicas\\ Córdoba, Argentina\\
%
\url{https://www.eco.unc.edu.ar/}\\%
\textit{ORCiD: \href{https://orcid.org/0000-0003-3297-3854}{0000-0003-3297-3854}}\\%
\href{mailto:ignacio.girela@unc.edu.ar}{\nolinkurl{ignacio.girela@unc.edu.ar}}%
}
