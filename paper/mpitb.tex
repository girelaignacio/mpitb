% !TeX root = RJwrapper.tex
\title{mpitb: A toolbox for calculating multidimensional poverty indices in R}


\author{by Ignacio Girela}

\maketitle

\abstract{%
(An abstract of less than 150 words.) This article presents mpitb, an integrated framework for calculating multidimensional poverty indices (MPIs). This package supports the popular Alkire-Foster framework for measuring multidimensional poverty.
}

\hypertarget{introduction}{%
\section{Introduction}\label{introduction}}

Although poverty is widely understood as the inability to achieve basic standards of living, the measurement of poverty has traditionally been based on monetary deficits. However, over the past two decades, there has been a significant increase in the use of multidimensional approaches to measuring poverty in an attempt to capture different aspects of this phenomenon. In this context, the measurement method proposed by Alkire and Foster (2011) (AF, hereafter) has become notably popular due to its flexibility and ability to inform policy. For instance, the well-known global Multidimensional Poverty Index (MPI), yearly published by the OPHI and the UNDP, is based on the AF method. Furthermore, many countries are developing their own MPIs to track poverty reduction progress using the AF method.

Although this method is increasingly popular, calculating AF measures in practice is still a challenging task due to the lack of a unified and integrated framework on how to produce, analyse and present different estimates. Motivated by this issue, OPHI has recently launched a Stata package (Suppa 2023) to provide {[}\ldots{]}. This package seeks to adapt this framework to R users.

Some R packages that computes AF measures include \CRANpkg{MPI} (Kukiattikun and Chainarong 2022) {[}\ldots{]}. Notwithstanding, these packages do not consider the survey structure from which household surveys affecting the estimation of standard errors and, consequently, affecting subsequent statistical inference.

This paper will first review some R packages on interactive graphics and their tooltip implementations. A new package \CRANpkg{ToOoOlTiPs} that provides customized tooltips for plot, is introduced. Some example plots will then be given to showcase how these tooltips help users to better read the graphics.

\hypertarget{measuring-multidimensional-poverty-the-alkire-foster-method}{%
\section{Measuring multidimensional poverty: the Alkire-Foster Method}\label{measuring-multidimensional-poverty-the-alkire-foster-method}}

\hypertarget{customizing-tooltip-design-with}{%
\section{\texorpdfstring{Customizing tooltip design with \pkg{ToOoOlTiPs}}{Customizing tooltip design with }}\label{customizing-tooltip-design-with}}

\pkg{ToOoOlTiPs} is a packages for customizing tooltips in interactive graphics, it features these possibilities.

\hypertarget{a-gallery-of-tooltips-examples}{%
\section{A gallery of tooltips examples}\label{a-gallery-of-tooltips-examples}}

The \CRANpkg{palmerpenguins} data (Horst, Hill, and Gorman 2020) features three penguin species which has a lovely illustration by Alison Horst in Figure \ref{fig:penguins-alison}.

\begin{figure}
\includegraphics[width=1\linewidth,height=0.3\textheight]{penguins} \caption{Artwork by \@allison\_horst}\label{fig:penguins-alison}
\end{figure}

Table \ref{tab:penguins-tab-static} prints at the first few rows of the \texttt{penguins} data:

\begin{table}

\caption{\label{tab:penguins-tab-static}A basic table}
\centering
\fontsize{7}{9}\selectfont
\begin{tabular}[t]{l|l|r|r|r|r|l|r}
\hline
species & island & bill\_length\_mm & bill\_depth\_mm & flipper\_length\_mm & body\_mass\_g & sex & year\\
\hline
Adelie & Torgersen & 39.1 & 18.7 & 181 & 3750 & male & 2007\\
\hline
Adelie & Torgersen & 39.5 & 17.4 & 186 & 3800 & female & 2007\\
\hline
Adelie & Torgersen & 40.3 & 18.0 & 195 & 3250 & female & 2007\\
\hline
Adelie & Torgersen & NA & NA & NA & NA & NA & 2007\\
\hline
Adelie & Torgersen & 36.7 & 19.3 & 193 & 3450 & female & 2007\\
\hline
Adelie & Torgersen & 39.3 & 20.6 & 190 & 3650 & male & 2007\\
\hline
\end{tabular}
\end{table}

Figure \ref{fig:penguins-ggplot} shows an plot of the penguins data, made using the \CRANpkg{ggplot2} package.

\begin{verbatim}
penguins %>% 
  ggplot(aes(x = bill_depth_mm, y = bill_length_mm, 
             color = species)) + 
  geom_point()
\end{verbatim}

\begin{figure}
\centering
\includegraphics{mpitb_files/figure-latex/penguins-ggplot-1.pdf}
\caption{\label{fig:penguins-ggplot}A basic non-interactive plot made with the ggplot2 package on palmer penguin data. Three species of penguins are plotted with bill depth on the x-axis and bill length on the y-axis. Visit the online article to access the interactive version made with the plotly package.}
\end{figure}

\hypertarget{summary}{%
\section{Summary}\label{summary}}

We have displayed various tooltips that are available in the package \pkg{ToOoOlTiPs}.

\hypertarget{references}{%
\section*{References}\label{references}}
\addcontentsline{toc}{section}{References}

\hypertarget{refs}{}
\begin{CSLReferences}{1}{0}
\leavevmode\vadjust pre{\hypertarget{ref-af11}{}}%
Alkire, Sabina, and J. E. Foster. 2011. {``Counting and Multidimensional Poverty Measurement.''} \emph{Journal of Public Economics} 95 (7): 476--87.

\leavevmode\vadjust pre{\hypertarget{ref-palmerpenguins}{}}%
Horst, Allison Marie, Alison Presmanes Hill, and Kristen B Gorman. 2020. \emph{{palmerpenguins}: Palmer Archipelago (Antarctica) Penguin Data}. \url{https://allisonhorst.github.io/palmerpenguins/}.

\leavevmode\vadjust pre{\hypertarget{ref-MPIpkg}{}}%
Kukiattikun, Kittiya, and Chainarong Chainarong. 2022. \emph{MPI: Computation of Multidimensional Poverty Index (MPI)}. \url{https://CRAN.R-project.org/package=MPI}.

\leavevmode\vadjust pre{\hypertarget{ref-mpitb_Stata}{}}%
Suppa, Nicolai. 2023. {``{m}pitb: A Toolbox for Multidimensional Poverty Indices.''} \emph{The Stata Journal} 23 (3): 625--57. \url{https://doi.org/10.1177/1536867X231195286}.

\end{CSLReferences}

\bibliography{RJreferences.bib}

\address{%
Ignacio Girela\\
CONICET - Universidad Nacional de Córdoba\\%
Facultad de Ciencias Económicas\\ Córdoba, Argentina\\
%
\url{https://www.eco.unc.edu.ar/}\\%
\textit{ORCiD: \href{https://orcid.org/0000-0003-3297-3854}{0000-0003-3297-3854}}\\%
\href{mailto:ignacio.girela@unc.edu.ar}{\nolinkurl{ignacio.girela@unc.edu.ar}}%
}
