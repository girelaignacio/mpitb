% !TeX root = RJwrapper.tex
\title{mpitb: A toolbox for calculating multidimensional poverty indices in R}


\author{by Ignacio Girela}

\maketitle

\abstract{%
This article presents \pkg{mpitb}, an R package for calculating multidimensional poverty indices (MPIs) based on the popular Alkire-Foster measurement approach. \pkg{mpitb} package provides a tractable and extensive framework for researchers, analysts, and practitioners working on mutltidimensional poverty measurement projects in the same vein of the well-known global MPI workflow.
}

\hypertarget{introduction}{%
\section{Introduction}\label{introduction}}

Although poverty is widely understood as the inability to achieve basic standards of living, the measurement of poverty has traditionally been based on monetary deficits. However, over the past two decades, there has been a significant increase in the use of multidimensional approaches to measuring poverty in an attempt to capture different dimensions of this phenomenon. In this context, the dual-cut-off-counting measurement approach proposed by Alkire and Foster (2011) (AF, hereafter) has become particularly popular due to its flexibility and ability to inform policy. For example, the AF method is the basis of the global Multidimensional Poverty Index (MPI) which is yearly published by the United Nations Development Programme and the Oxford Poverty and Human Development Initiative (UNDP and OPHI 2022). In addition, many countries are developing their own official MPIs to track poverty reduction progress using the AF method.

This measurement approach stands out both for providing an extensive policy information platform and remaining relatively understandable by the general public. In other words, AF measures point estimates entail simple algebraic operations. Some R packages that computes AF measures include \CRANpkg{MPI} (Kukiattikun and Chainarong 2022) and \CRANpkg{mpindex} (Abdulsamad 2023). Notwithstanding, these packages do not account for the complex survey design of micro data and, as a result, assume the data was obtained through a simple random sampling which is barely the case of household surveys. Certainly, not considering the complex survey design affects further statistical inference exercises of interest. Namely, is multidimensional poverty in one subgroup of the population greater than another? Has poverty been reduced over time?

Therefore, calculating AF measures in practice is still a challenging task Motivated by this problem, OPHI launched a Stata package (Suppa 2023) with the aim of providing a framework that mirrors the production process of the well-known global MPI to researchers, analysts and practitioners. To do so, Suppa (2023) (continue\ldots).
This package adapts this framework for R users.

\hypertarget{measuring-multidimensional-poverty-the-alkire-foster-method}{%
\section{Measuring multidimensional poverty: the Alkire-Foster Method}\label{measuring-multidimensional-poverty-the-alkire-foster-method}}

\hypertarget{overview-of}{%
\section{\texorpdfstring{Overview of \pkg{mpitb}}{Overview of }}\label{overview-of}}

\pkg{mpitb} is a packages for \ldots{}

\hypertarget{applications}{%
\section{Applications}\label{applications}}

Suppa (2023) provide a set of examples \ldots. With the view of consistency, we apply \ldots{} on the same data\ldots{}

\begin{verbatim}
library(mpitb)

data <- swz_mics14
\end{verbatim}

Table \ref{tab:penguins-tab-static} prints at the first few rows of the \texttt{penguins} data:

\begin{table}

\caption{\label{tab:penguins-tab-static}A basic table}
\centering
\fontsize{7}{9}\selectfont
\begin{tabular}[t]{l|l|r|r|r|r|l|r}
\hline
species & island & bill\_length\_mm & bill\_depth\_mm & flipper\_length\_mm & body\_mass\_g & sex & year\\
\hline
Adelie & Torgersen & 39.1 & 18.7 & 181 & 3750 & male & 2007\\
\hline
Adelie & Torgersen & 39.5 & 17.4 & 186 & 3800 & female & 2007\\
\hline
Adelie & Torgersen & 40.3 & 18.0 & 195 & 3250 & female & 2007\\
\hline
Adelie & Torgersen & NA & NA & NA & NA & NA & 2007\\
\hline
Adelie & Torgersen & 36.7 & 19.3 & 193 & 3450 & female & 2007\\
\hline
Adelie & Torgersen & 39.3 & 20.6 & 190 & 3650 & male & 2007\\
\hline
\end{tabular}
\end{table}

Figure \ref{fig:penguins-ggplot} shows an plot of the penguins data, made using the \CRANpkg{ggplot2} package.

\begin{verbatim}
penguins %>% 
  ggplot(aes(x = bill_depth_mm, y = bill_length_mm, 
             color = species)) + 
  geom_point()
\end{verbatim}

\begin{figure}
\centering
\includegraphics{mpitb_files/figure-latex/penguins-ggplot-1.pdf}
\caption{\label{fig:penguins-ggplot}A basic non-interactive plot made with the ggplot2 package on palmer penguin data. Three species of penguins are plotted with bill depth on the x-axis and bill length on the y-axis. Visit the online article to access the interactive version made with the plotly package.}
\end{figure}

\hypertarget{summary}{%
\section{Summary}\label{summary}}

In this paper, we introduced a package for calculating multidimensional poverty indices based on the Alkire-Foster method

\hypertarget{acknowledgements}{%
\section{Acknowledgements}\label{acknowledgements}}

I want to thank

\hypertarget{references}{%
\section*{References}\label{references}}
\addcontentsline{toc}{section}{References}

\hypertarget{refs}{}
\begin{CSLReferences}{1}{0}
\leavevmode\vadjust pre{\hypertarget{ref-mpindexpkg}{}}%
Abdulsamad, Bhas. 2023. \emph{{m}pindex: Multidimensional Poverty Index (MPI)}. \url{https://CRAN.R-project.org/package=mpindex}.

\leavevmode\vadjust pre{\hypertarget{ref-af11}{}}%
Alkire, Sabina, and J. E. Foster. 2011. {``Counting and Multidimensional Poverty Measurement.''} \emph{Journal of Public Economics} 95 (7): 476--87.

\leavevmode\vadjust pre{\hypertarget{ref-MPIpkg}{}}%
Kukiattikun, Kittiya, and Chainarong Chainarong. 2022. \emph{MPI: Computation of Multidimensional Poverty Index (MPI)}. \url{https://CRAN.R-project.org/package=MPI}.

\leavevmode\vadjust pre{\hypertarget{ref-mpitb_Stata}{}}%
Suppa, Nicolai. 2023. {``{m}pitb: A Toolbox for Multidimensional Poverty Indices.''} \emph{The Stata Journal} 23 (3): 625--57. \url{https://doi.org/10.1177/1536867X231195286}.

\leavevmode\vadjust pre{\hypertarget{ref-gMPI-report2022}{}}%
UNDP, and OPHI. 2022. {``\href{}{2022 Global Multidimensional Poverty Index (MPI)}.''} \emph{UNDP (United Nations Development Programme)}.

\end{CSLReferences}

\bibliography{RJreferences.bib}

\address{%
Ignacio Girela\\
CONICET - Universidad Nacional de Córdoba\\%
Facultad de Ciencias Económicas\\ Córdoba, Argentina\\
%
\url{https://www.eco.unc.edu.ar/}\\%
\textit{ORCiD: \href{https://orcid.org/0000-0003-3297-3854}{0000-0003-3297-3854}}\\%
\href{mailto:ignacio.girela@unc.edu.ar}{\nolinkurl{ignacio.girela@unc.edu.ar}}%
}
